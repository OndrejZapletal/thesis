%% Název práce:
\nazev{Image Recognition by Convolutional Neural Networks - Basic Concepts}
{Rozpoznávání obrazů konvolučními neuronovými sítěmi - základní koncepty}
% Jméno a příjmení autora ve tvaru
\autor[Bc.]{Ondřej}{Zapletal}

%% Jméno a příjmení vedoucího/školitele včetně titulů
\vedouci[Ing.]{Karel}{Horák}[Ph.D.]

%% Jméno a příjmení oponenta včetně titulů
%  [tituly před jménem]{Křestní}{Příjmení}[tituly za jménem]
% Pokud nemá titul za jménem, smažte celý řetězec '[...]'
% Uplatní se pouze v prezentaci k obhajobě;
% v případě, že nechcete, aby se na titulním snímku prezentace zobrazoval oponent, pouze jej zakomentujte;
% u obhajoby semestrální práce se oponent nezobrazuje
\oponent[doc.\ Mgr.]{Křestní}{Příjmení}[Ph.D.]

%% Označení oboru studia
\oborstudia{Cybernetics, Control and Measurements}{Kybernetika, automatizace a měření}
%% Označení fakulty
\fakulta{Faculty of Electrical Engineering and Communication}{Fakulta elektrotechniky a komunikačních technologií}

%% Označení ústavu
\ustav{Department of Control and Instrumentation}{Ústav automatizace a měřicí techniky}

\logofakulta[loga/FEKT_zkratka_barevne_PANTONE_CZ]{loga/UTKO_color_PANTONE_CZ}


%% Rok obhajoby
\rok{Rok}
\datum{1.\,1.\,1970} % Datum se uplatní pouze v prezentaci k obhajobě

%% Místo obhajoby
% Na titulních stránkách bude automaticky vysázeno VELKÝMI písmeny
\misto{Brno}

%% Abstrakt
\abstrakt{Abstrakt práce v~originálním jazyce
}{Překlad abstraktu v~angličtině (nebo češtině pokud je originální jazyk angličtina)
}

%% Klíčová slova
\klicovaslova{Klíčová slova v~originálním jazyce}%
	{Překlad klíčových slov v~angličtině nebo češtině}

%% Poděkování
\podekovanitext{Rád bych poděkoval vedoucímu diplomové práce panu Ing.~XXX YYY, Ph.D.\ za odborné vedení, konzultace, trpělivost a podnětné návrhy k~práci.}